\chapter{翻译注意事项}

\section{正文}

\textbf{标点符号}

译稿中的标点符号要遵循中文表达习惯和中文标点符号的使用习惯,不能照搬原文的标点符号。

使用中文标点,对比中文英文括号:
\begin{itemize}
    \item (英文)():!?,.
    \item (中文)():!?,。
\end{itemize}

\textbf{字体}

\vspace{4mm}
\begin{tabular}{ll}
    \toprule
    pdf                          &  latex code                         \\
    \midrule
    正文为宋体                   &  \verb!正文为宋体!                  \\
    \textit{英文斜体对应楷体}    &  \verb!\textit{英文斜体对应楷体}!   \\
    \textbf{加粗对应黑体}        &  \verb!\textbf{加粗对应黑体}!       \\
    \texttt{等宽字体对应仿宋}    &  \verb!\texttt{等宽字体对应仿宋}!   \\
    \bottomrule
\end{tabular}
\vspace{4mm}

\textbf{专业术语}

对于专业术语的翻译可以参考全国科技名词委的\href{http://www.termonline.cn/index.htm}{术语在线},
,虽然绝大多数原书中出现的术语并没有被收录。

关于英文术语的表述。英文术语首次出现时,应该根据该术语的流行情况,优先使用简写形式,
并在其后使用括号加英文、中文全称注解,格式为(举例):HTML(Hypertext
Markup Language,超文本标识语言)。
然后在下文中直接使用简写形式。当然,必要时也可以根据语境使用中、英文全称。

\paragraph{参考}

\begin{itemize}
\item \href{http://www.cmpbook.com/index.php?id=131}{机械工业出版社作译者须知}.
\item \href{http://www.ituring.com.cn/article/501527}{图灵技术图书译者须知}.
\end{itemize}

\section{数学}

更多使用的教程参考 \href{https://en.wikibooks.org/wiki/LaTeX/Mathematics}{\LaTeX/Mathematics wiki}

\subsection{公式编号}

行内数学环境: $a + b$ , \verb!$a + b$! 或者 \verb!\(a + b\)!。

多行数学环境使用 \verb!$$a + b$$!,  \verb!\[a + b\]!。
$$a + b$$

有编号的公式使用 \texttt{euqation} 环境:

\begin{equation}\label{euqation:one_plus_one}
    1 + 1 = \lambda f . \lambda x . f\, x
\end{equation}


需要子编号的公式使用 \texttt{subequations} 环境:
\begin{subequations}
    \begin{align} % if you need alignment
        a &= b \\
        a &= b = c
    \end{align}
    \begin{gather}
        a = b\\
        a = b = c
    \end{gather}
    \begin{equation}
        a = b
    \end{equation}
    \begin{equation}
        a = b = c
    \end{equation}
\end{subequations}

推荐使用 \texttt{align} 而不是 \texttt{eqnarray},bug 更少。
如果不需要编号,在环境名后面加上星号即可,比如: \verb!\begin{equation*} \end{equation*}!。

\subsection{字体}

数学字体

\begin{tabular}{ll}
$\mathrm{JABCDEabcde1234}$      &  \verb!$\mathrm{JABCDEabcde1234}$!     \\
$\mathit{JABCDEabcde1234}$      &  \verb!$\mathit{JABCDEabcde1234}$!     \\
$\mathsf{JABCDEabcde1234}$      &  \verb!$\mathsf{JABCDEabcde1234}$!     \\
$\mathtt{JABCDEabcde1234}$      &  \verb!$\mathtt{JABCDEabcde1234}$!     \\
$\mathnormal{JABCDEabcde1234}$  &  \verb!$\mathnormal{JABCDEabcde1234}$! \\
$\mathcal{JABCDEabcde1234}$     &  \verb!$\mathcal{JABCDEabcde1234}$!    \\
$\mathscr{JABCDEabcde1234}$     &  \verb!$\mathscr{JABCDEabcde1234}$!    \\
$\mathfrak{JABCDEabcde1234}$    &  \verb!$\mathfrak{JABCDEabcde1234}$!   \\
$\mathbb{JABCDEabcde1234}$      &  \verb!$\mathbb{JABCDEabcde1234}$!     \\
\end{tabular}

如果在数学环境的中途需要使用文字, 使用 \verb!\text{}!.

\subsection{定理与证明}

\begin{theorem}[稳定性]\label{theorem:stability}
 如果 $\Gamma \vdash_{\mathcal{R}} J$ 则 $\Gamma \vdash_{\mathcal{R} \cup \mathcal{R}'} J$
\end{theorem}

(找不到原书使用的 J 是是什么字体, 可能是 J 去掉了前面的勾)

在 \texttt{local.tex} 中定义了定理(\texttt{theorem})、引理(\texttt{lemma})、
推论(\texttt{corollary})、定义(\texttt{definition})的环境。请使用这些环境。


证明需要使用 \texttt{proof} 环境,在证明结束后会有一个方框。
\begin{proof}
    显然。
\end{proof}


\subsection{宏}

\LaTeX 可以使用 \verb!\newcommand\commandname{commandbody}! 来定义新的命令。
如果 commandname 已经被定义,使用 \verb!renewcommand! 覆盖原有定义。

\texttt{local.tex} 文件中定义了一些宏。

\begin{verbatim}
\newcommand{\Comm}[1]{\mbox{\textit{#1}}}
\end{verbatim}

\texttt{[1]} 表示接受一个参数, \texttt{\#1}用来引用这个参数. 参数通过 \texttt{\{\}} 传递。

例如: $\Comm{add}$  $\qquad$ \vspace{1cm} \verb!$\Comm{add}$!

可以在文中随时定义新的宏,在使用后删除宏,避免污染环境。

\newcommand{\aaaa}{aaaa}
\aaaa
\let\aaaa\undefined

在 \texttt{bcprules.sty} 文件中定义了用于排版 inference rules 的宏。


\section{引用}

在需要引用的公式,章节,定理后面加上 \verb!\label{label_name}!, 使用 \verb!\ref{label_name}!
引用。名字最好有意义。涉及章节的引用先空着,用注释说明。

定理\ref{theorem:stability}, 公式 \ref{euqation:one_plus_one}。

注意 \verb!\label{}! 不要换号,需要和被引用的目标在同一行。

\section{索引}

原书最后有一个索引(index),列出了所有术语以及出现的页数。这里我们打算参考
\href{https://github.com/exacity/deeplearningbook-chinese}
{exacity/deeplearningbook-chinese} 的做法,做一个术语表(glossary)。术语表相比索引的好处
是能够表达中英文术语对照, 而且能够保证对术语翻译的一致性。

每个术语项使用 \verb!\makeglossaries! 定义, \texttt{name} 域写术语的中文翻译,\texttt{description}
域写术语的英文。

\begin{verbatim}
\newglossaryentry{latex}
{
    name = 拉泰赫,
    description = {\LaTeX}
}
\end{verbatim}

在正文中通过 \verb!\gls{}! 的方式引用, 比如 \gls{a-equiv}。 所有术语都在 \texttt{terminology.tex}
中定义。在定义之前先检查是否已经被定义。

关于 \texttt{glossaries} 包的更多说明参考
\href{https://www.sharelatex.com/learn/Glossaries}{Glossaries | ShareLaTeX} 或者在
命令行执行 \texttt{texdoc glossaries} 查看文档。
