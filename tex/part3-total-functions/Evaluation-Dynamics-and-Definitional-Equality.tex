\section{动态求值和定义相等}

EF中求值断言的归纳定义由下面的规则给出:
\begin{subequations}
    \begin{gather}
        \dfrac{}
            {\mathsf{lam}\{\tau\}(x.e) \Downarrow \mathsf{lam}\{\tau\}(x.e)} \\
        \dfrac{ e_1 \Downarrow \mathsf{lam}\{\tau\}(x.e) \quad [e_2/x]e \Downarrow v}
            {\mathsf{ap}(e_1;e_2)\Downarrow v}
    \end{gather}
\end{subequations}
显然地,如果$e \Downarrow v$,那么$v\ \mathsf{val}$,
并且如果$e\ \mathsf{val}$,那么$e \Downarrow e$。

\begin{theorem}{}
$e \Downarrow v\ \ iff.\ \ e \Arrow^* v 并且v\ \mathsf{val}$。
\end{theorem}

\begin{proof}
充分性的证明按照规则(8.6)进行归纳,使用类似Theorem(7.2)的证明方法。

必要性的证明按照规则(5.1)进行归纳,证明依赖于Lemma(7.4)的一个类比,
该引理指出求值在逆向执行下是封闭的,这是通过对规则(8.5)的归纳得到的。
\end{proof}

EF的动态按名调用的定义相等是由下面的规则的定义的。

\begin{subequations}
    \begin{gather}
        \dfrac{}
            {\Gamma \vdash \mathsf{ap} ( \mathsf{lam} \{\tau\}(x.e_2);e_1) \equiv [e_1/x]e_2:\tau_2 } \\
        \dfrac{\Gamma \vdash e_1 \equiv e'_1:\tau_2 \Arrow \tau 
                \quad \Gamma \vdash e_2 \equiv e'_2:\tau_2}
            {\Gamma \vdash \mathsf{ap}(e_1;e_2) \equiv \mathsf{ap}(e'_1;e'_2):\tau} \\
        \dfrac{ \Gamma,x:\tau_1 \vdash e_2 \equiv e'_2:\tau_2 }
            {\Gamma \vdash \mathsf{lam}\{\tau_1\}(x.e_2) 
                \equiv \mathsf{lam}\{tau_1\}(x.e'_2):\tau_1 \Arrow \tau_2 }
    \end{gather}
\end{subequations}

按值调用的定义相等需要更多的机制。
主要的思想是通过限制规则(8.7a)使得参数必须是值。
此外,必须扩展值使其包含变量,因为在按值调用中,函数用参数变量代表参数的值。
按值调用的定义相等断言采用下面的形式:
$$\Gamma \vdash e_1 \equiv e_2 : \tau,$$
其中$\Gamma$由成对的假设$x:\tau$组成,
例如$x\ \mathsf{val}$声明了对范围内的每一个变量$x$,其类型是值。
$\Gamma \vdash e\ \mathsf{val}$表示在这些假设下$e$是值,
所以有$x:\tau$,$x\ \mathsf{val} \vdash x\ \mathsf{val}$