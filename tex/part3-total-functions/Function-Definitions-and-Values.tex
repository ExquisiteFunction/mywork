\chapter{函数定义和值}

在\textbf{E}语言中,我们可以进行一些计算,例如加倍一个给定的表达式,但是我们还不能表达加倍这个概念本身。
为了描述将一个数字加倍的模式,我们从加倍特定的数中抽象出使用\textit{变量}代表固定但不确定的数,来表达加倍任意一个数。
加倍的任何实例可以通过将变量替换为数字表达式来获得。
一般来说,一个表达式可能涉及多个变量,因此我们需要指定哪些变量在特定的上下文中变化,从而调用关于这些变量的\textit{函数}。

在这一章中,我们将讨论\textbf{E}语言中函数的两个扩展。首先,显然的,需要在语言中加入\textit{函数定义}。
一个函数由 绑定到一个带有绑定变量的抽象绑定树的一个名字定义, 其中绑定变量作为这个函数的参数。
一个函数通过将绑定变量替换为特定表达式来被\textit{应用},得到一个表达式。

被函数的域和范围被限定为nat类型和str类型,因为只有这两种类型的表达式。
这种函数被称为\textit{一阶函数},与之相对的是\textit{高阶函数},其允许参数和结果是其他函数。
由于函数的域和范围是类型,这要求我们引入\textit{函数类型},其元素是函数。
因此,我们可以构造\textit{高阶类型}的函数,其域和范围本身也是函数类型。


\subimport{./}{First-Order-Functions}
\subimport{./}{Higher-Order-Functions}
\subimport{./}{Evaluation-Dynamics-and-Definitional-Equality}
\subimport{./}{Dynamic-Scope}
\subimport{./}{Notes-and-Exercises}

%\Downarrow
















