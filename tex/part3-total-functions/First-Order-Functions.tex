\section{一阶函数}
\textbf{ED}语言从\textbf{E}语言按照如下的语法扩展了函数的定义和函数的应用:
$$
\begin{array}{llll}
\mathsf{Exp}\quad e \quad ::= \quad &\mathsf{apply}\{f\}(e)    &f(e)    &\text{应用}\\
                                    &\mathsf{fun}\{\tau_1;\tau_2\}(x_1.e_2;f.e)    &\mathsf{fun} f(x_1:\tau_1):\tau_2 = e_2  \mathsf{\ in\ }  e    &\text{定义}
\end{array}
$$
表达式$\mathsf{fun} \{\tau_1;\tau_2\}({x_1}.{e_2};f.e)$将函数名$f$和$e$绑定到模式$x_1.e_2$,其中$x_1$是参数,$e_2$是定义。
函数的域和范围分别是$\tau_1$类型和$\tau_2$类型。
表达式$\mathsf{apply} \{f\}(e)$实例化了$f$和$e$的绑定。
\textbf{ED}的静态语义定义了两种断言形式:
\begin{definition}
表达式类型,$e:t$,声明了$e$有$t$类型
\end{definition}
\begin{definition}
函数类型$f(\tau_1):\tau_2$,声明了$f$是一个参数类型为$\tau_1$并且返回值类型为$\tau_2$的函数。
\end{definition}

断言$f(\tau_1):\tau_2$被称为$f$的\textit{函数头},它指明了一个函数的域类型和范围类型。

\textbf{ED}的静态语义由下面的规则定义:
\begin{subequations}
    \begin{gather}
        \dfrac{\Gamma,x_1:\tau_1\vdash e_2:\tau_2  \quad  \Gamma,f(\tau_1):\tau_2\vdash e:\tau}
            {\Gamma \vdash \mathsf{fun}\{\tau_1;\tau_2\}(x_1.e_2;f.e):\tau} \\
        \dfrac{\Gamma \vdash f(\tau_1):\tau_2  \quad  \Gamma\vdash e:\tau_1}
            {\Gamma \vdash \mathsf{apply}\{f\}(e):\tau_2}
    \end{gather}
\end{subequations}

\textit{函数替换},写作$[[x.e / f]]e'$,由对e'的结构进行归纳定义,和普通的替换一样。
但是,一个函数名$f$并不代表一个表达式,除了在应用形式$\mathsf{apply}\{f\}(e)$中。
函数替换由下面的规则定义:
\begin{equation}
    \dfrac{}
        {[[x.e/ f ]]\mathsf{apply}\{f\}(e') = \mathsf{let}([[x.e/f]]e';x.e)}
\end{equation}
在$f$和参数$e'$的应用处,函数替代产生一个let表达式,将$x$绑定到$f$和$e'$的任何其他应用的扩展结果上。
    
\begin{lemma}
 如果 $\Gamma,f(\tau_1):t_2 \vdash e:\tau \ and\ \Gamma,x_1:\tau_1 \vdash e_2:_t2$,
 那么 $\Gamma \vdash [[x_1.e_2/f]]e:\tau$。
\end{lemma}
    
\begin{proof}
通过对第一个前提的规则归纳,类似于引理4.4的证明。
\end{proof}

ED的动态语义则通过函数替换来定义:
\begin{equation}
    \dfrac{}
        {\mathsf{fun}\{\tau_1;\tau_2\}(x_1.e_2;f.e) \Arrow [[x_1.e_2/f]]e}
\end{equation}
由于函数替换通过let表达式替代了$f$的所有应用,因此就没必要给出应用表达式的规则了
(本质上,他们的行为就像求值过程中的被替换的变量,而不是语言中的原始操作。)

通过一些努力,可以将\textbf{ED}的安全性从高阶函数的安全性理论中推导出来,这正是我们的接下来要讨论的。